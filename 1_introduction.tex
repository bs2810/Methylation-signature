\chapter{Introduction}
% Byrjá á genome og genetic variations
    % Færa yfir í að ekki sé hægt að útskýra allar breytingar og sjúkdóma og því sé verið að færa í aukana að sequenca methylation status (epimutations)
    % Einn svoleiðis sjúkdómur er Kabuki syndrome - Búið að finna einhver gen en nú er talið að methylation munstrið spili enn mikilvægara hlutverk.

\section{Epimutations}
The completion of the Human Genome Project, has provided us with a nearly complete list of genes needed to produce a human. However, of equal importance is a second system that cells use to determine when and where a particular gene will be expressed during development. This system is overlaid on DNA in the form of epigenetic marks that are heritable during cell division but do not alter the DNA sequence \cite{robertson2005dna}. 

As with genes, these epigenetic marks can be mutated causing the genomic locus to deviate significiantly from the normal state. Those epimutations can be classified into two main types: primary epimutations that are thought to represent stochastic errors in the establishment or maintenance of an epigenetic state, while secondary epimutations are downstream events related to an underlying change in DNA sequence \cite{horsthemke2006epimutations}. The most common epigenetic modification of DNA (epimutation) in mammals is a methylation of the C5 position of cytosine in CpG dinucleotides, referred to as 5mC methylation. 

With recent dramatic advances in genomic technologies, genome-wide surveys of cohorts of patients with certain disorders or diseases for point mutations and structural variations have greatly advanced the understanding of their genetic etiologies. However, even after whole-genome-sequencing (WGS), no causative mutation can be identified in many such cases \cite{barbosa2018identification}. % laga citation

\section{Outline}
The goal of this thesis is to define the methylation signature of Kabuki syndrome using Oxford nanopore methylation data and determine the heritability of the methylation pattern.